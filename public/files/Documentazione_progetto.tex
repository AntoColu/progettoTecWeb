\documentclass[12pt,a4paperS]{report}
\usepackage[italian]{babel}
\usepackage[utf8]{inputenc}
\usepackage{graphicx}
\usepackage{natbib}
\usepackage{float}
\usepackage{hyperref}

\graphicspath{ {./img/} }

\begin{document}
	\begin{titlepage}
		\begin{center}
			
			\Large
			\textbf{Progetto Tecnologie Web 2023}
			
			\vspace{0.5cm}
			
			Documentazione del progetto sviluppato per il corso di Tecnologie Web
			
			\vspace{0.5cm}
			
			Anno scolastico: 2022/2023
			
			\vspace{1.5cm}
			
			\textbf{Gruppo 58}
			
			\vspace{0.5cm}
			
			\textbf{Antonio Colucci}
			
			\vspace{0.5cm}
			
			\textbf{Matricola: 1087022}
			
			\vspace{2cm}
			
			\includegraphics[width=5.5cm]{logo_uni}
			
			\vspace{1.5cm}
			
			\Large
			Corso di Ingegneria Informatica e dell'Automazione presso l'Università Politecnica delle Marche
		\end{center}
	\end{titlepage}
	\tableofcontents
	\newpage
	
	\hypertarget{descrizione}{\chapter{Descrizione del sito}}
	\label{descrizione}
		\begin{normalsize}
			Il sito sviluppato è dedicato ad un'azienda di noleggio auto.
			L'azienda offre appunto servizi di noleggio di svariati modelli di auto suddivisi per categoria:
			\begin{itemize}
				\item Auto piccole: sono automobili compatte, ottime per l'utilizzo in città e per piccoli spostamenti;
				\item Auto medie: sono automobili non troppo grandi, che permettono un buon utilizzo cittadino, ma anche possibilità di spostamenti più lunghi;
				\item Auto grandi: sono automobili comode e con qualche accorgimento tecnologico in più. Sono ottime per affrontare lunghi spostamenti ed offrono maggiore spazio per i bagagli;
				\item SUV: sono auto grandi e alte, che permettono una visuale migliore alla guida ed un comfort di altissimo livello.
			\end{itemize}
			Il sito offre quattro livelli di utenza, suddivisi in base a diverse funzionalità:
			
			\section{Utente pubblico}
				Gli utenti pubblici posso accedere alle funzionalità base del sito, che sono comuni a tutti i livelli di utenza.
				\newline
				Nel dettaglio essi possono: reperire informazioni sull'azienda e sulla sua posizione; reperire dettagli sui servizi offerti; ottenere maggiori informazioni consultando le FAQ; accedere al catalogo delle auto che l'azienda mette a disposizione per il noleggio; accedere al sito con le proprie credenziali, oppure effettuare la registrazione di un nuovo account.
				
			\section{Utente registrato (Cliente)}
				I clienti sono gli utenti che hanno effettuato la registrazione al sito, quindi possiedono un proprio account.
				\newline
				Oltre a poter usufruire di tutte le funzionalità di un utente pubblico (eccezione fatta per la registrazione di un nuovo account), i clienti possono: noleggiare un'automobile per un determinato periodo di tempo scelto in fase di noleggio; modificare i dati personali nella propria area riservata.
			
			\section{Membro dello staff aziendale}
				Quest'area è dedicata ai membri dello staff aziendale ovvero a tutti i dipendenti, i quali possono: accedere al catalogo auto per andare ad inserire, modificare o eliminare le auto già presenti in catalogo; visualizzare un elenco delle auto noleggiate e dei rispettivi clienti, in un determinato mese scelto all'inizio.
			
			\section{Amministratore}
				L'amministratore dell'azienda può effettuare tutte le operazioni consentite agli utenti pubblici ed allo staff, inoltre può: gestire tutto lo staff (creazione, modifica ed eliminazione dei membri); cancellare gli utenti registrati; aggiornare le FAQ del sito; accedere ad un'area dove può ottenere, per l'anno corrente, il numero di auto noleggiate in base al mese scelto.
		\end{normalsize}
		
	\hypertarget{diagrammi}{\chapter{Diagrammi e schemi del Sito}}
	\label{diagrammi}
		\begin{normalsize}
			\section{Schemi dei link}
				Di seguito vengono riportati gli schemi dei link, divisi per livello di utenza, i quali forniscono una visione completa di tutti i collegamenti delle pagine del sito.
				\begin{figure}[H]
					\centering
					\includegraphics[width=1.15\textwidth, height=1.15\textheight, trim=100 0 0 0,keepaspectratio]{Grafici/Link_sez_pubblica.png}
					\newline
					\caption{Schema dei link della sezione pubblica}
				\end{figure}
				\begin{figure}[H]
					\centering
					\includegraphics[width=1.15\textwidth, height=1.15\textheight, trim=100 0 0 0,keepaspectratio]{Grafici/Link_sez_clienti.png}
					\caption{Schema dei link della sezione clienti}
				\end{figure}
				\begin{figure}[H]
					\centering
					\includegraphics[width=1.15\textwidth, height=1.15\textheight, trim=100 0 0 0,keepaspectratio]{Grafici/Link_sez_staff.png}
					\newline
					\caption{Schema dei link della sezione staff}
				\end{figure}
				\begin{figure}[H]
					\centering
					\includegraphics[width=1.15\textwidth, height=1.15\textheight, trim=100 0 0 0,keepaspectratio]{Grafici/Link_sez_admin.png}
					\caption{Schema dei link della sezione admin}
				\end{figure}
				
			\section{Database}
				Il database è composto da 4 tabelle: una dedicata alla categoria rappresentante un gruppo di auto; una dedicata alle auto; una per la FAQ; una per gli utenti.
				\newline
				Per quanto riguarda la relazione tra Auto e Categoria, ogni auto appartiene ad una ed una sola categoria, mentre ogni categoria può avere diverse automobili.
				\newline
				Parlando invece della relazione tra Auto ed Utente, ogni utente può noleggiare più auto, ma in periodi diversi, ed ugualmente un'auto può essere noleggiata da più utenti in periodi diversi.
				\subsection{Schema E-R}
					\begin{figure}[H]
						\centering
						\includegraphics[width=1.15\textwidth, height=1.15\textheight, trim=100 0 0 0,keepaspectratio]{Grafici/Database.png}
						\newline
						\caption{Schema E-R del DB}
					\end{figure}
		\end{normalsize}
		
	\hypertarget{mockup}{\chapter{Mockup}}
	\label{mockup}
	\begin{normalsize}
		In questo capitolo verranno inserite tutte le immagini del mockup realizzato.
		
		\section{Sezione pubblica}
			Le immagini seguenti rappresentano tutte le pagine di cui si compone la sezione pubblica del sito.
			\subsection{Homepage}
				La homepage è la medesima per gli utenti di tutti i livelli.
				\newline
				L'unica differenza si trova nella barra di navigazione, perchè per ogni livello di utenza la barra fornirà delle funzioni diverse.
				\begin{figure}[H]
					\centering
					\includegraphics[width=0.95\textwidth, height=0.95\textheight, keepaspectratio]{Mockup/Homepage.png}
					\caption{Homepage pubblica}
				\end{figure}
				
			\subsection{Catalogo noleggi}
				Il catalogo delle auto è lo stesso sia per gli utenti registrati che non.
				\newline
				Inoltre sono presenti dei campi compilabili per filtrare le auto in base alla categoria, fascia di prezzo e numero di posti.
				\begin{figure}[H]
					\centering
					\includegraphics[width=0.91\textwidth, height=0.91\textheight, keepaspectratio]{Mockup/Catalogo.png}
					\caption{Catalogo noleggi}
				\end{figure}
			
			\subsection{Dettagli auto selezionata}
				La pagina riguardante i dettagli dell'auto selezionata nel catalogo è la medesima sia per gli utenti registrati che non.
				\newline
				Ricordo che il noleggio effettivo dell'auto può essere eseguito solo dagli utenti registrati, infatti qualora un utente non registrato cliccasse sul tasto "Noleggia", verrebbe reindirizzato alla pagina di login.
				\begin{figure}[H]
					\centering
					\includegraphics[width=0.85\textwidth, height=0.85\textheight, keepaspectratio]{Mockup/Dettagli_auto.png}
					\caption{Dettagli auto}
				\end{figure}
			
			\subsection{Contatti}
				La pagina riguardante i contatti dello sviluppatore è identica sia per gli utenti di livello 0 che di livello 1.
				\begin{figure}[H]
					\centering
					\includegraphics[width=1\textwidth, height=1\textheight, keepaspectratio]{Mockup/Contatti.png}
					\caption{Contatti}
				\end{figure}
				
			
			\subsection{Login}
				\begin{figure}[H]
					\centering
					\includegraphics[width=1\textwidth, height=1\textheight, keepaspectratio]{Mockup/Login.png}
					\caption{Form per il login}
				\end{figure}
			
			\subsection{Registrazione}
				\begin{figure}[H]
					\centering
					\includegraphics[width=1\textwidth, height=1\textheight, keepaspectratio]{Mockup/Registrazione.png}
					\caption{Form per la registrazione di un nuovo utente}
				\end{figure}
		
		\newpage
		\section{Sezione clienti}
			Le immagini seguenti rappresentano tutte le pagine di cui si compone la sezione riservata agli utenti registrati al sito.
			
			\subsection{I miei noleggi}
				\begin{figure}[H]
					\centering
					\includegraphics[width=1\textwidth, height=1\textheight, keepaspectratio]{Mockup/Storico_noleggi_utente.png}
					\caption{Storico delle auto noleggiate dal cliente loggato}
				\end{figure}
			
			\subsection{Area riservata}
				Area riservata del cliente, contenente tutti i suoi dati personali e con la possibilità di modificarli.
				\begin{figure}[H]
					\centering
					\includegraphics[width=1\textwidth, height=1\textheight, keepaspectratio]{Mockup/Area_riservata.png}
					\caption{Area riservata con i dati personali del cliente}
				\end{figure}
			
			\subsection{Modifica dati personali}
				\begin{figure}[H]
					\centering
					\includegraphics[width=1\textwidth, height=1\textheight, keepaspectratio]{Mockup/Modifica_dati.png}
					\caption{Form di modifica dei dati personali}
				\end{figure}
			
		\newpage
		\section{Sezione staff}
			Le immagini seguenti rappresentano tutte le pagine di cui si compone la sezione riservata ai membri dello staff.
			
			\subsection{Gestione auto}
				Qui troviamo un'area dedicata alla gestione delle auto, ovvero permette l'inserimento, la modifica e l'eliminazione di una o più automobili.
				\newline
				Questa pagina è usata anche dall'admin.
				\begin{figure}[H]
					\centering
					\includegraphics[width=0.75\textwidth, height=0.75\textheight, keepaspectratio]{Mockup/Gestione_auto.png}
					\caption{Pagina per le gestione delle auto}
				\end{figure}
			
			\subsection{Nuova auto}
				Pagina che permette l'inserimento di una nuova auto nel catalogo, fornendo tutti i relativi dati e delle immagini che riprendono l'auto dall'esterno.
				\newline
				Questa pagina è usata anche dall'admin.
				\begin{figure}[H]
					\centering
					\includegraphics[width=0.82\textwidth, height=0.82\textheight, keepaspectratio]{Mockup/Nuova_auto.png}
					\caption{Form di inserimento di una nuova auto}
				\end{figure}
			
			\subsection{Modifica auto}
				La pagina dedicata alla modifica dei dati di un'auto selezionata è identica alla pagina per l'inserimento di una nuova auto.
				\newline
				Questa pagina è usata anche dall'admin.
			
			\subsection{Storico dei noleggi}
				Qui lo staff può scegliere un mese e visualizzare l'elenco delle automobili noleggiate quel mese.
				\newline
				Questa pagina è usata anche dall'admin.
				\begin{figure}[H]
					\centering
					\includegraphics[width=0.88\textwidth, height=0.88\textheight, keepaspectratio]{Mockup/Storico_noleggi_staff.png}
					\caption{Storico dei noleggi di un dato mese}
				\end{figure}
				
		\section{Sezione admin}
			Le immagini seguenti rappresentano tutte le pagine di cui si compone la sezione riservata all'amministratore dell'azienda.
			
			\subsection{Gestione Staff}
				Area dedicata alla gestione dello staff, dove è possibile inserire, modificare ed eliminare uno o più membri dello staff.
				\begin{figure}[H]
					\centering
					\includegraphics[width=0.86\textwidth, height=0.86\textheight, keepaspectratio]{Mockup/Gestione_staff.png}
					\caption{Gestione dei membri dello staff}
				\end{figure}
			
			\subsection{Inserimento membro staff}
				\begin{figure}[H]
					\centering
					\includegraphics[width=1\textwidth, height=1\textheight, keepaspectratio]{Mockup/Nuovo_staff.png}
					\caption{Form di inserimento di un nuovo membro dello staff}
				\end{figure}
			
			\subsection{Modifica dati membro staff}
				Questa pagina è esteticamente identica alla pagina che permette l'inserimento di un nuovo membro dello staff, ma senza i campi "username" e "password".
			
			\subsection{Gestione Clienti}
				Pagina dove l'admin può ottenere tutti i clienti del sito ed eliminarne uno o più di uno tramite il pulsante presente su ogni card.	
				\begin{figure}[H]
					\centering
					\includegraphics[width=1\textwidth, height=1\textheight, keepaspectratio]{Mockup/Gestione_clienti.png}
					\caption{Gestione dei clienti}
				\end{figure}
		
			\subsection{Gestione FAQ}
				Qui è possibile gestire le FAQ, andando ad inserire, modificare ed eliminare una o più FAQ.
				\begin{figure}[H]
					\centering
					\includegraphics[width=1\textwidth, height=1\textheight, keepaspectratio]{Mockup/Gestione_faq.png}
					\caption{Gestione delle FAQ}
				\end{figure}
			
			\subsection{Inserimento nuova FAQ}
				\begin{figure}[H]
					\centering
					\includegraphics[width=1\textwidth, height=1\textheight, keepaspectratio]{Mockup/Nuova_faq.png}
					\caption{Form di inserimento di una nuova FAQ}
				\end{figure}
			
			\subsection{Modifica FAQ}
				Questa pagina è esteticamente identica alla pagina che permette l'inserimento di una nuova FAQ.
			
			\newpage
			\subsection{Statistiche}
				In questa pagina l'amministratore può ottenere un prospetto di quante auto sono state noleggiate in un mese scelto dell'anno corrente.
				\begin{figure}[H]
					\centering
					\includegraphics[width=1\textwidth, height=1\textheight, keepaspectratio]{Mockup/Statistiche.png}
					\caption{Numero di auto noleggiate in un certo mese}
				\end{figure}
			
	\end{normalsize}
	
	\hypertarget{soluzioni}{\chapter{Soluzioni adottate}}
	\label{soluzioni}
	\begin{normalsize}
		\begin{itemize}
			\item Nel catalogo sono stati inseriti filtri riguardanti la categoria, la fascia di prezzo (prezzo minimo e prezzo massimo) ed il numero di posti delle auto.
			\newline
			Tutti questi filtri lavorano in logica AND e vengono ricoperti tutti i casi di completamento.
			\item Manipolazione dei nomi da assegnare alle immagini (ed i relativi path) attraverso tagli e concatenazioni di stringhe, in modo tale da rispettare uno standard per tutte le immagini delle auto.
			\newline
			Il suddetto standard è: \texttt{<Marca><Modello>\_img<Lato auto>.jpg}
			\newline
			Se la marca e/o il modello dell'auto presentano degli spazi, questi ultimi vengono eliminati.
			\item In fase di noleggio vengono eseguiti tutti i controlli necessari sulle date inserite e sulla disponibilità dell'auto, ovvero: 
				\begin{itemize}
					\item le date di inizio e fine noleggio devono essere specificate;
					\item la data di inizio noleggio deve essere precedente a quella di fine;
					\item la data di inizio deve essere uguale o successiva alla data odierna;
					\item se l'auto è libera allora può essere noleggiata;
					\item se l'auto è occupata allora la data di inizio noleggio deve essere successiva alla data di consegna dell'auto da parte dell'utente che attualmente la possiede.
				\end{itemize}
				    
		\end{itemize}
	\end{normalsize}
\end{document}